\documentclass[a4paper]{article}

%% Language and font encodings
\usepackage[english]{babel}
\usepackage[utf8x]{inputenc}
\usepackage[T1]{fontenc}

%% Sets page size and margins
\usepackage[a4paper,top=3cm,bottom=2cm,left=3cm,right=3cm,marginparwidth=1.75cm]{geometry}

%% Useful packages
\usepackage{amsmath}
\usepackage{graphicx}
\usepackage[colorinlistoftodos]{todonotes}
\usepackage[colorlinks=true, allcolors=blue]{hyperref}

\usepackage{float}
\restylefloat{table}

\setlength{\parindent}{0cm}

\title{Machine Learning Course}
\author{Wellington Ricardo Pinheiro}

\begin{document}
\maketitle

\section{Intro}

Machine Learning is the science of getting computers to learn, without being explicitly programmed.

\medskip

There are two approaches to work with ML:

\begin{itemize}
\item Supervised Learning: where a set of right answers are given is given to the algorithm
\item Unsupervised Learning: where a data set is given and the algorithm finds some structure among the data
\end{itemize}

\subsection{Supervised Learning}

When we have a data set with right answers and we want to know the answer for some known property of this data set. For example a \textbf{House Price Prediction} or a \textbf{Weather Prediction} based on past information.

\medskip

There can be two approaches to solve a problem using a Supervised Learning algorithm:

\begin{itemize}
\item \textbf{Regression} problem: predict continuous valued output
\item \textbf{Classification}: Discrete valued output. Work with discrete f(x) values (x can be continuous)
\end{itemize}

A Supervised Learning Algorithm can be seen as a function $f$ such as:

\[f(x_1, x_2, ..., x_k) = y\]

Where:

\begin{itemize}
\item $x_i$, para $1 \le i \le k$, is a \textbf{feature}
\item $y$ is the valued output
\end{itemize}

\subsection{Unsupervised Learning Algorithm}

When we have a data set without answers and we want to discover some structure among the items. Each structural set found is named \textbf{cluster}. The Unsupervised Learning Algorithm is also known as a \textbf{clustering algorithm}.

\section{Linear Regression}

Given a data set with right answers, a linear regression is a way to approximate a result based on a straight line that best fits the data set. It's basically a regression problem.

\medskip

Remember:

\begin{itemize}
\item \textbf{Supervised Learning}: given the "right answer" for each example in the data
\item \textbf{Regression Problem}: Predict real-valued output
\item \textbf{Classification}: Discrete-valued output
\end{itemize}

\begin{table}[]
\centering
\begin{tabular}{l|r}
\textbf{Size(m2)} & \textbf{Price in 1000's} \\\hline
2104     & 460 \\
1416     & 232 \\
1534     & 315 \\
852      & 178
\end{tabular}
\caption{Training set}
\label{tab:training-set-example}
\end{table}

Table~\ref{tab:training-set-example} shows a distribution that relates the size of houses in $feet^2$ that maps to their prices in $1000$s of dollars).

Based on the training set we can define:

\begin{itemize}
\item $m$: number of training examples
\item $x$: input variable or feature
\item $y$: output variable or target variable
\item $(x,y)$: one training example
\item $(x^{(i)}, y^{(i)})$: the $i_{th}$ training example
\end{itemize}

Let's also define $h$ as the \textbf{hypothesis}. That is, $h$ is a function that maps from $x's$ to a estimated value. In the example of Table~\ref{tab:training-set-example} \textbf{$h$ is a hypothesis that maps from the size of a house to an estimated price}

\subsection{How to represent $h$}

$h$ can be represented as a linear polymon in the form:
\[h_\theta(x) = \theta_0 + \theta_1 * x\]

$h(x)$ is a shorthand for $h_\theta(x)$.

\medskip

This approach to solve problem is known as \textbf{Linear regression with one variable} or \textbf{univariate linear regression}.

\section{Cost Function}

Hypothesis: $h_\theta(x) = \theta_0 + \theta_1 x$, where:

\begin{itemize}
\item{$\theta_i's$: parameters}
\end{itemize}

How to choose $\theta_i's$?

\medskip

Choose $\theta_0$, $\theta_1$ so that $h_\theta(x)$ is close to $y$ for our training examples $(x,y)$. To do this we define a cost function $J(\theta_0, \theta_1)$, also known as \textbf{Squared Error Function}, as:

\[J(\theta_0, \theta_1) = \frac{1}{2m}\sum_{i=1}^m(h_\theta x^{(i)}-y^{(i)})^2\]

Minimizing $J(\theta_0, \theta_1)$ will give the $\theta_0$ and $\theta_1$ that makes $h_\theta(x)$ close to training examples.

\section{Cost Function Intuition}
\begin{itemize}
\item Hypothesis: $h_\theta(x) = \theta_0 + \theta_1 * x$
\item Parameters: $\theta_0, \theta_1$
\item Cost function: $J(\theta_0,\theta_1) = \frac{1}{2m}\sum_{i=1}^{m}(h_\theta(x^{(i)}) - y^{(i)})^2$
\item Goal: minimize $J(\theta_0, \theta_1)$
\end{itemize}

Let's start with a simplified version of $h_\theta = \theta_1 * x$. Then we're going to minimize $J(\theta_1)$.

\begin{table}[]
\centering
\begin{tabular}{l|r}
\textbf{x} & \textbf{y} \\\hline
1     & 2 \\
2     & 2 \\
3     & 3
\end{tabular}
\caption{Cost function training set}
\label{tab:cost-function-training-set}
\end{table}

Consider the training set presented in Table~\ref{tab:cost-function-training-set}. Based on in, we can use a $\theta_1 = 1$, then:

$$
J(\theta_1) = \frac{1}{2m}\sum_{i=1}^{m}(h_\theta(x^{(i)}) - y^{(i)})^2
= \frac{1}{2m}\sum_{i=1}^{m}(\theta_1 * x^{(i)} - y^{(i)})^2 = \frac{1}{2m}(0^2 + 0^2 + 0^2) = 0^2$$

\section{Gradient Descent}

Have some function $J(\theta_0,\theta_1)$
Want to min $J(\theta_0,\theta_1)$

\textbf{Outline}
\begin{itemize}
\item Start with some $\theta_0, \theta_1$
\item Keep changing $\theta_0, \theta_1$ to reduce $J(\theta_0,\theta_1)$ until we hopefully end up at minimum
\end{itemize}

\subsection{Gradient Descent Algorithm}

repeat until convergence (for $j=0$ and $j=1$ - simultaneously update $\theta_0$ and $\theta_1$) \{
$$
\theta_j := \theta_j - \alpha \frac{\partial}{\partial \theta_j}J(\theta_0, \theta_1)
$$
\}

\medskip

where:

\begin{itemize}
\item $\alpha$: is the learning rate. The bigger it is the more aggressive the approximation occurs.
\end{itemize}

Simultaneous update of $\theta_0$ and $\theta_1$ means:

\begin{itemize}
\item $temp_0 := \theta_0 - \alpha \frac{\partial}{\partial \theta_j}J(\theta_0, \theta_1)$
\item $temp_0 := \theta_1 - \alpha \frac{\partial}{\partial \theta_j}J(\theta_0, \theta_1)$
\item $\theta_0 := temp0$
\item $\theta_1 := temp1$
\end{itemize}

\subsection{Gradient Descent Intuition}

Remember that:

\medskip

repeat until convergence (for $j=0$ and $j=1$ - simultaneously update $\theta_0$ and $\theta_1$) \{
$$
\theta_j := \theta_j - \alpha \frac{\partial}{\partial \theta_j}J(\theta_0, \theta_1)
$$
\}

Where:

\begin{itemize}
\item $\alpha$ is the learning rate
\item $\frac{\partial}{\partial \theta_j}J(\theta_0,\theta_1)$ is the derivative term
\end{itemize}

Again, let's simplify and minimize $J(\theta_1)$, with $\theta_1 \in \rm I\!R$

If $\alpha$ is too small, gradient descent can be slow.  But if $\alpha$ is too large, the gradient descent can overshoot the minimum. It may fail to converge, or even diverge.

Note that as each iteration of the algorithm approaches  the local minimum, the derivative term becomes smaller. This way it's not needed to update $\alpha$ value over time.

\section{Gradient Descent for Linear Regression}

Idea: apply gradient descent to the cost function $J(\theta_0, \theta_1)$. Then:

$$
\frac{\partial}{\partial \theta_j}J(\theta_0,\theta_1) \frac{1}{2m}\sum_{i=1}^{m}(h_\theta(x^{(i)}) - y^{(i)})^2
$$

Dividing in cases for $j = 1$ and $j = 2$:

\begin{itemize}
\item $j = 0: \frac{\partial}{\partial \theta_0}J(\theta_0,\theta_1) = \frac{1}{m}\sum_{i=1}^m (h_0(x^{(i)}) - y^{(i)})$
\item $j = 1: \frac{\partial}{\partial \theta_1}J(\theta_0,\theta_1) = \frac{1}{m}\sum_{i=1}^m (h_0(x^{(i)}) - y^{(i)}) x^{(i)}$
\end{itemize}

\medskip

repeat until convergence \{
$$
\theta_0 := \theta_0 - \alpha \frac{\partial}{\partial \theta_0}J(\theta_0,\theta_1) = \theta_0 - \alpha \frac{1}{m}\sum_{i=1}^m (h_0(x^{(i)}) - y^{(i)})
$$
$$
\theta_1 := \theta_1 - \alpha \frac{\partial}{\partial \theta_1}J(\theta_0,\theta_1) = \theta_1 - \alpha \frac{1}{m}\sum_{i=1}^m (h_0(x^{(i)}) - y^{(i)}) x^{(i)}
$$
\}

About the convergence of $J(\theta_0,\theta_1)$ it's important to say that $J$ is a \textbf{convex function}, a bowl shaped function that has a global optimum point.

\subsection{Batch Gradient Descent}

Each step of gradient descent uses all the training examples this is why we call it \textbf{Batch}. There are some cases which not all training examples are needed.

\section{Liner Regression with Multiple Features}

In previous chapters the linear regression dealt with only one feature. In this chapters is presented how this approach also works with multiple features. Table~\ref{tab:multiple-features-example} shows an example of multiple features that influences the price of a house.

\begin{table}[]
\centering
\begin{tabular}{|r|r|r|r|r|}
\textbf{Size(feet$^2$)} & \textbf{Number of bedrooms} & \textbf{Number of floors} & \textbf{Age of home (years)} & \textbf{Price (\$1000\'s)} \\ \hline
2104 & 5 & 1 & 45 & 460 \\
1416 & 3 & 2 & 40 & 232 \\
1534 & 3 & 2 & 30 & 315 \\
852  & 2 & 1 & 36 & 178
\end{tabular}
\caption{Multiple Features (variables)}
\label{tab:multiple-features-example}
\end{table}

Let's define:

\begin{itemize}
  \item n: number of features
  \item $x^{(i)}$: input (features) of $i^{th}$ training example
  \item $x_{j}^{(i)}$: value of feature $j$ in $i^{th}$ training example
\end{itemize}

The hypothesis using multiple features is defined as:

$$
h_\theta = \theta_0 + \theta_1 x_1 + \theta_2 x_2 + \cdots + \theta_n x_n
$$

Example: $h_\theta = 80 + 0.1 x_1 + 0.01 x_2 + 3 x_3 - 2 x_4$

For convenience we define $x_0 = 1$, so it's possible to rewrite $h_\theta$ as:

$$
h_\theta = \theta_0 x_0 + \theta_1 x_1 + \theta_2 x_2 + \cdots + \theta_n x_n
$$

$X$ and $\theta$ can be represented as matrices:

\medskip

\begin{center}
\begin{tabular}{cc}
$X = \left[ \begin{array}{c}
x_0 \\ x_1 \\ x_2 \\ \cdots \\ x_n \end{array} \right]$ &
$\theta = \left[ \begin{array}{c}
\theta_0 \\ \theta_1 \\ \theta_2 \\ \cdots \\ \theta_n \end{array} \right]$
\end{tabular}
\end{center}

Based on the the matrices $X$ and $\theta$ $h_\theta(x)$ can be written as:

\[ h_\theta(x) = \theta^T x \]

\subsection{Gradient Descent for Multiple Variables}

For multiple variables the cost function $J$ is defined as: $J(\theta_0, \theta_1, \cdots, \theta_n) = \frac{1}{2m}\sum_{i=1}^{m}(h_\theta(x^{(i)})-y^{(i)})^2$

Instead of writing $J(\theta_0, \theta_1, \cdots, \theta_n)$ will be only used $J(\theta)$.

Gradient Descent

repeat (\{
$$
\theta_j := \theta_j - \alpha \frac{\partial}{\partial \theta_j}J(\theta)
$$
\} (simultaneously update for every $j = 0, 1, \cdots, n$)

Applying the partial derivative

\medskip

repeat (\{
$$
\theta_j := \theta_j - \alpha \frac{1}{m}\sum_{i=1}{m}(h_\theta(x^{(i)})-y^{(i)})x_{j}^{(i)}
$$
\} (simultaneously update for every $j = 0, 1, \cdots, n$)

\subsection{Feature Scaling}

Make sure features are on similar scale.

Ex: $x_1 = $ size $(0-2000 feet^2)$ and $x_2 = $ number of bedrooms $(1-5)$.

If scales are very different it'll take a long time to gradient descent finds the local optimum. The approach here is to scale the features.

\medskip

Get every feature into approximately a $-1 < x_i < 1$ range by dividing each feature by its maximum value. That is:

\medskip

\begin{center}
\begin{tabular}{cc}
$x_1 = \frac{size (feet^2)}{2000}$ &
$x_2 = \frac{number of bedrooms}{5}$
\end{tabular}
\end{center}

\subsection{Mean Normalization}

Replace $x_i$ with $x_i - \mu_i$ to make features have approximately zero mean (do not apply to $x_0 = 1$). Also divide $x_i - \mu_i$ by the max value minus the min value for the feature. The formula is: $$\frac{x_i - \mu_i}{S_i}$$ where $S_i$ is the max $-$ min value for feature $i$.

\subsection{The Learning Rate}

Convergence test: when $J(\theta)$ decreases less than $10^{-3}$ it could be considered that $\theta$ is an answer.

Summary about learning rate:

\begin{itemize}
\item if $\alpha$ is too slow there will be a slow convergence
\item if $\alpha$ is too large $J(\theta)$ may not decrease on every iteration; may not converge
\end{itemize}

\subsection{Features and Polynomial Regression}

House price prediction

\[h_\theta(x) = \theta_0 + \theta_1 * frontage + \theta_2 * depth \]

in the previous expression $frontage$ is the like $x_1$ and $depth$ is the $x_2$. In fact, it's possible to create and use other features in the linear regression. For example, a new feature $area = frontage * depth$ can be used and the hypothesis will be $h_\theta(x) = \theta_0 + \theta_1 x$. Sometimes using new feature can lead to better models.

\subsubsection{Polynomial Regression}

Depending on how your data is distributed, other model than linear can fit better. For example, can be used a quadratic distribution with a hypothesis $h_\theta(x) = \theta_0 + \theta_1 x + \theta_2 x^2$ or a cubic $h_\theta(x) = \theta_0 + \theta_1 x + \theta_2 x^2 + \theta_3 x^3$.

For example, when dealing with a feature $size$ related to the size of a house, the hypothesis can be defined as $h_\theta(x) = \theta_0 + \theta_1 (size) + \theta_2 (size)^2 + \theta_3 (size)^3$, then defining $x_1 = size$, $x_2 = (size)^2$ and $x_3 = (size)^3$. Doing this it's possible to use the already known liner regression mechanics for $h_\theta(x) = \theta_0 + \theta_1 x_1 + \theta_2 x_2 + \theta_3 x_3$. \textbf{A important note} is that in this case feature scaling really matters here.

Other models than can also be used, for example, $h_\theta(x) = \theta_0 + \theta_1 size + \theta_2 \sqrt{size}$.

\section{Normal Equation}

Normal equation is a method to solve $\theta$ analytically.

For example, to find a local minimum for a function $J(\theta) = a\theta^2 + b\theta + c$ is sufficient to find a value for $\theta$ so that $\frac{d}{d\theta}J(\theta) = 0$.

In a linear regression problem we have $J(\theta_0, \theta_1, \cdots, \theta_n)$ then we have to find $\theta$ set so that $\frac{\partial}{\partial\theta_j}J(\theta) = 0$ for every $j$.

As example, suppose data in Table~\ref{tab:multiple-features-example}. Let's add a column on left representing a $x_0$ feature with all values 1. Then we represent this table as a matrix $X$:

\begin{center}
\begin{tabular}{cc}
$X = \left[ \begin{array}{ccccc}
1 & 2104 & 5 & 1 & 45 \\
1 & 1416 & 3 & 2 & 40 \\
1 & 1534 & 3 & 2 & 30 \\
1 & 852  & 2 & 1 & 36
\end{array} \right]$ &
$y = \left[ \begin{array}{c}
460 \\ 232 \\ 315 \\  178 \end{array} \right]$
\end{tabular}
\end{center}

Based on $X$ and $y$ we can solve $\theta$ as:

$$
\theta = (X^T X)^{-1} X^T y
$$

Features don't need to be scaled when Normal Equation is used.

\subsection{When use gradient descent or normal equation}

\begin{table}[H]
\centering
\begin{tabular}{|l|l|}
\hline
\textbf{Gradient Descent} & \textbf{Normal Equation} \\
\hline
Need to choose $\alpha$ & Don't need to choose $\alpha$ \\
Needs many iterations & Solve analytically (no iterations) \\
Works well even when $n$ is large & Need to compute $(X^T X)^{-1}$ (can be slow for $n\times n$ matrices) \\
& Slow if $n$ is very large (consider large $> 10^4$) \\
\hline
\end{tabular}
\caption{Gradient Descent vs. Normal Equation}
\label{tab:gradient-descent-normal-equation}
\end{table}

\section{Classification - Logistic Regression}

email: spam / not spam
online transactions: fraudulent (yes/no)
tumar: malignat / benignant

$y \in {0, 1}$ where $0$ is negative and $1$ is positive.

\subsection{Hypothesis Representation}

The logistic regression model we want should be $0 < h(\theta) < 1$.

$h(\theta) = g(\theta^T x)$ and $g(z) = \frac{1}{1 + e^{-z}}$ (Sigmoid function or logistic function). Putting the two equations together then we have:

\[h_\theta(x) = \frac{1}{1 + e^{-\theta^T x}}\]


Interpretation of Hypothesis Output:

$h_\theta(x)$ = estimated probability that $y = 1$ on input $x$.

Example: if $x = \left[ \begin{array}{c} x_0 \\ x_1 \end{array} \right] =
\left[ \begin{array}{c} 1 \\ tumorSize \end{array} \right]$

\newline

$h_theta(x) = 0.7$: tell patient that $70\%$ chance of tumor being malignant.

We can also define $h_\theta(x) = P(y=1 | x;\theta)$: probability that $y=1$, given $x$, parameterized by $\theta$.




\section{Quiz Tips}

\subsection{Week 1 - Introduction}

\begin{itemize}
\item In the first question is important to observe that experience E \textbf{is the data set}, the information the algorithm uses to learn. The task T \textbf{is the action performed by the algorithm}, i.e., the task of predicting the Weather, or a Disease, or a House Price. Each of these predictions is based on features. The performance P \textbf{is the output given by the algorithm}.
\item For questions two and three remember that when the algorithm output domain is a limited discrete set then the problem can be a classification problem, otherwise it's a regression problem. Be aware when the output domain set in discrete but large, that in these cases the output domain can be seen as a real domain and not discrete.
\item In question four it's important to note that when clusters are mentioned or characteristics that are not previously known then it's talking about unsupervised learning.
\end{itemize}

\subsection{Week 1 - Linear Regression with One Variable}

\begin{itemize}
\item TODO
\end{itemize}

\section{References}

\begin{itemize}
\item Wiki for Week 1: \url{
https://share.coursera.org/wiki/index.php/ML:Main\#Week\_1}
\end{itemize}
\end{document}
